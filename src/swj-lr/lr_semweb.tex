The representation formalisms and technologies that make up the Semantic Web can be used to enable interoperability for language databases on a wide scale. The purpose of Linked Data is to enable structured data to be shared on the Web. The Linked Open Data paradigm sets out four rules for representation of web resources \citep{bernersLee2006_linkeddata}: 

\begin{enumerate}
\item referred entities should be designated by using Uniform Resource Identifiers (URIs),
\item these URIS should be resolvable over HTTP,
\item data should be represented by means of specific W3C standards (such as RDF),
\item and a resource should include links to other resources. 
\end{enumerate}

These rules facilitate data interoperability in that they require that entities are addressed in a globally unambiguous way (1), that they can be accessed (2) and interpreted (3), and that entities that are associated on a conceptual level are also physically associated with each other (4) \cite{ChiarcosLOD}. An essential part of Linked Data is the Resource Description Framework (RDF), a language that was developed to represent information, such as metadata, about resources on the Web. In an RDF graph, information is expressed with subject-predicate-object triples. Each component of the triple is encoded by a URI, making it globally unambiguous on the Web. These resources can also link to other resources in a standardized way, so that various RDF graphs can share links and be combined into larger graphs. SPARQL is the query language for RDF data and it also consists of triple patterns, with additional optional patterns and conjunctions, disjunctions, etc. \cite{prud2008sparql}. One tool that instantiates the SPARQL protocol is the SPARQL endpoint. A SPARQL endpoint is a service that  supports querying RDF data from a single graph over HTTP and it may also be used to query multiple distributed RDF graphs via endpoints. Thus attain federated query across multiple resources.
