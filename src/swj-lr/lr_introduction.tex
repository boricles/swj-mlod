%
%Richard's hackday project aimed at querying the LLOD via ISO 639-3
%code. At that time, these queries worked (see below for several of
%them). Now they don't seem to return anything (hence I haven't
%included them in the MLODE hack day results). There might be some
%fairly obvious reason for them not working, but I'm not sure what it
%is. I do remember that at one point the URL or the endpoint or
%something was changed for some reasons (hosting was with Pablo's
%group?).
%I do think though that the ability to query LLOD for resources via
%language code, etc., is a very, very strong selling point for the data
%post-proceedings and for the LLOD cloud in general. We aren't going to
%get any serious users (do we currently have any users?) beyond the
%Leipzig group (and perhaps some others), if it's not functioning (or
%if we can't explain how to use it since what was once working no
%longer works).
%
%It would be great to get such "exploratory" data queries working and I
%know Richard and a few others want to submit something to the data
%post-proceedings on exactly that.

%ENDPOINT: http://mlode-sparql.nlp2rdf.org/sparql

The amount of computational linguistic resources available has grown considerably in recent years. 
This is true both of primary resources - audio corpora, dictionaries, ontologies - and secondary resources - parsers, segmenters, WordNets. However, limited interoperability and licensing constraints between primary resources are major obstacles for data use and reuse. Interoperability can be either structural, apparent when similar formalisms are used to access data, or conceptual, where the resources themselves share a common vocabulary \citep{ide-pustejovsky2010-interoperability}. Representing data so that they conform to the paradigms and languages of the Semantic Web fosters interoperability for resources, as well as providing an infrastructure with a vibrant development community that can be used to query these resources. 

The Linguistics Linked Open Data (LLOD) cloud is a (sub-)cloud of the Linked Open Data cloud, consisting of databases of linguistic corpora or metadata that conforms to the Linked Open Data paradigm \citep{bernersLee2006_linkeddata}. This cloud is being developed by the Open Linguistics Working Group (OWLG) \cite{owlg4lrec}, an open community dedicated towards creating and developing standards and linguistics resources for public use. The LLOD already contains dozens of publicly available online databases. Here, we briefly outline the technologies behind the LLOD (described at length elsewhere, cf. \cite{ldl-llod, ChiarcosLOD}), and, for the first time, provide a SPARQL endpoint to the LLOD, along with example queries of potential use to linguistics researchers. We show two queries with their results, the first displaying all of the databases within the LLOD cloud which have information related to the language-specific ISO code for an example language (each database can then be queried either individually or together). The second query gathers all of the typological data available through the endpoint from the World Atlas of Language Structures (WALS) \citep{Haspelmath_etal2008} typological database, also for the same example language. We conclude by briefly discussing possible applications of querying the LLOD cloud. 