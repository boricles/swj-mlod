The amount of linguistic resources available on the Web has grown considerably in recent years. This is true of both data resources, such as lexical data, multimedia recordings and annotated corpora, as well as for computational tools for cleaning and analysis, such as chunkers, part of speech taggers and parsers. However, limited interoperability between data formats and data licensing constraints are major obstacles for data use, reuse and sharing. Representing data so that they conform to the Semantic Web framework fosters interoperability of resources, as well as providing an infrastructure with a vibrant development community that can be used to query these resources. Linked Data refers to Semantic Web framework best practices for publishing and connecting structured data. One initiative to share openly available data published in Linked Data is called Linked Open Data.

The Linguistic Linked Open Data (LLOD) cloud\footnote{\url{http://linguistics.okfn.org/resources/llod/}} is a sub-cloud of the Linked Open Data cloud,\footnote{\url{http://richard.cyganiak.de/2007/10/lod/}} consisting of data from databases of linguistic corpora and metadata that conforms to the Linked Open Data paradigm \citep{bernersLee2006_linkeddata}. This cloud is being developed by the Open Linguistics Working Group (OWLG) \cite{owlg4lrec}, an open community that aims to promote open data in linguistics and that facilitates communication between researchers from different scientific disciplines and communities. The LLOD  contains data from dozens of publicly available online databases that have been converted to RDF. Here, we briefly outline the technologies behind the LLOD (described at length elsewhere, cf. \cite{ldl-llod, ChiarcosLOD}), and, for the first time we discuss accessing data in the LLOD through a SPARQL endpoint set up during the Multilingual Linked Open Data for Enterprises workshop. We provide a couple of examples of how to use the endpoint to query across resources linked in the LLOD. The first query displays all resources within the cloud that have information related to a language-specific ISO 639-3 three letter language name identifier. Once data sources that contain information about a given language are identified, each data source can then be further queried for detailed information regarding that given language. Our second example query identifies the typological features of a given language that are available in the World Atlas of Language Structures (WALS) \citep{Haspelmath_etal2008}. We conclude by briefly discussing the possible applications of querying the LLOD cloud for linguistic analysis and its potential use to language researchers. 
