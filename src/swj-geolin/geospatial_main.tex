% Journal:
%   Journal of Ambient Intelligence and Smart Environments (JAISE), IOS Press
%   Web Intelligence and Agent Systems: An International Journal (wias)
%   Semantic Web: Interoperability, Usability, Applicability (SW)
% Latex 2e
% Test file iosart2c.tex

%[seceqn,secfloat,secthm,crcready]

% options: wias, jaise, sw
\documentclass{iosart2c}


\usepackage[T1]{fontenc}
\usepackage{times}%

\usepackage{natbib}
%\usepackage[dvips]{hyperref}
\usepackage{amsmath}
\usepackage{dcolumn}
%\usepackage{endnotes}
\usepackage{graphics, url}


\newcolumntype{d}[1]{D{.}{.}{#1}}


\firstpage{1} \lastpage{5} \volume{1} \pubyear{2009}


\begin{document}
\begin{frontmatter}                           % The preamble begins here.

% CHANGE AS SUITED - RL.
%\pretitle{Pretitle}
\title{Enhancing Linguistic Resources with Geospatial Mapping %\thanks{Footnote in title.}
}

\runningtitle{Enhancing Linguistic Resources with Geospatial Mapping}
%\subtitle{Subtitle}

%\review{Name Surname, University, Country}{Name Surname, University, Country}{Name Surname, University, Country}


\author[A]{\fnms{Steven} \snm{Moran}%\thanks{Corresponding author. E-mail: editorial@iospress.nl.}%\thanks{Do not use capitals for the author's surname.}
},
\author[B,C]{\fnms{Richard} \snm{Littauer}}
and
\author[D]{\fnms{Boris} \snm{Villazon-Terrazas}}
\runningauthor{Moran, Littauer and Villazon-Terrazas}
\address[A]{Research Unit Quantitative Language Comparison, Ludwig Maximilian University, Geschwister Scholl Platz 1, D-80539 Munich, Germany\\ 
E-mail: bambooforest@gmail.com} %Steve? Probably not the right email. 


%\address[A]{Journal Production Department, IOS Press, Nieuwe Hemweg 6b, 1013 BG, Amsterdam,\\ The Netherlands\\ E-mail: first@somewhere.com}
%\address[B]{Department first, then University or Company name, Insert a complete correspondence (mailing) address, Abbreviate US states, Include country\\ E-mail: \{second,third\}@somewhere.com}
\address[B]{Department of Intelligent Computer Systems, University of Malta, Msida, MSD2080, Malta}
\address[C]{Computational Linguistics Department, Saarland University, Saarbr\"ucken, 66121, Germany\\  E-mail: littauer@coli.uni-saarland.de}
\address[D]{Intelligent Software Components, iSOCO, S.A., Av. del Partenon 16-18, Madrid, Spain\\
E-mail: bvillazon@isoco.com}

\begin{abstract}
The Linguistics Linked Open Data cloud, created and maintained by the Open Linguistics Working Group, is a (sub-)cloud of the Semantic Web, conforming to the Linked Open Data paradigm. The potential of a very large, interlinking, interoperable ontology for linguistics research is great; however, first adopters may be hesitant to upload their datasets or use the cloud, due to a large learning curve or a lack of obvious uses. Here, we present a research workflow, from a spreadsheet to RDF to visualisation, going through the entire iterative and incremental linked data cycle. We use only freely accessible technologies in the Semantic Web framework, as well as a dataset of lexical and geospatial information of Dogon languages in West Africa. We also present a visualisation of language data from the World Atlas of Language Structures using an endpoint within the LLOD. By doing so, we shed light upon the possibilities of the Semantic Web, and in particular the LLOD, for potential researchers in the digital humanities and computational sciences. %144 words
%The abstract should be clear, descriptive, self-explanatory and no longer than 200 words. It should alsobe suitable for publication in abstracting services. Do not include references or formulae in the abstract.
\end{abstract}

\begin{keyword}
Semantic Web\sep Linked Data\sep LLOD\sep Linguistics\sep Typology\sep Language Resources\sep Geospatial Mapping
%Keyword one\sep keyword two\sep keyword three\sep keyword four\sep keyword five
\end{keyword}

\end{frontmatter}

%http://www.slideshare.net/boricles/linguistic-resources-enhanced-with-geospatial-information

\section{Introduction}

Linked Data presents many opportunities to access and share data in different formats and for different purposes. In linguistics and related fields like cultural archaeology and population genetics, visualization of data points on maps is particularly beneficial in formulating hypotheses about data sets, particularly sparse ones, which is often the case in these fields. In this short report, we describe how we converted a spreadsheet that contains information about endangered Dogon languages and where they are spoken in small rural villages in Mali, West Africa, into an Resource Description Framework (RDF) triple store so that we could leverage other RDF tools to visualize these data. The result gives researchers a clearer picture of the dispersal of Dogon speakers and neighboring languages and we show that the spreadsheet-to-RDF conversion pipeline that we develop is applicable to any data set that can be combined with GPS coordinates.


\section{Related Work}
%I'm note entirely sure of much related work, to be honest. Uhm. Yeah. Boris? Steve?

%% Shouldn't we be describing the Dogon dataset, somewhere?

Regarding language visualisation, there has been some work on displaying language differences on a broad scale. This includes work presenting hierarchical or cross-linguistic data\cite{Rohrdantz:2012, RMB+10}, displaying related languages (gathered from WALS) by geographical proximity and relatedness \cite{LittEACL}, displaying word meaning with a world map \cite{theron}, and displaying language locations on a globe \cite{Haspelmath_etal2008}. However, the authors are not aware of any work using maps derived from data stored in RDF as here, nor of visualisations aimed at displaying language relatedness using lexical items and location together. %Right?

\section{Spreadsheet to geospatial information}
%http://www.slideshare.net/boricles/linguistic-resources-enhanced-with-geospatial-information
In this section we present the specialisation of the Linked Data Life Cycle presented in \cite{Villazon_2011} applied to linguistics resources enhanced to geospatial information.

\subsection{Linguistics Resources}\label{sec:lr}

Our initial data source consisted of a spreadsheet containing GPS and lexical information for .... %STEVE FILL OUT 

\subsection{URI design}
All the resources in our dataset are defined using a URI\footnote{\url{http://tools.ietf.org/html/rfc3986}} (Uniform Resource Identifier). URIs have been designed with simplicity, stability and manageability in mind, following common guidelines for their effective use\footnote{\url{http://www.w3.org/TR/cooluris/}}, \footnote{\url{http://www.w3.org/Provider/Style/URI}}, \footnote{\url{http://www.w3.org/TR/chips/}}.

\subsubsection{Base URI structure}

\surl{http://linguistic.linkeddata.es/}

\subsubsection{Vocabulary elements}

\surl{http://linguistic.linkeddata.es/ontology/{property|class}}

For example

\surl{http://linguistic.linkeddata.es/ontology/officialName}

\subsubsection{Instances}

\surl{http://linguistic.linkeddata.es/dataset/resource/{r. type|r. name}}

For example

\surl{http://linguistic.linkeddata.es/mlode/Village/Sokoura}



\subsection{Modelling}
The development of the linguistics vocabulary, which covers the linguistic information stored in the resources described in section \ref{sec:lr}, 
has been performed following an iterative approach
based on the reuse of existing knowledge resources. In a nutshell we have reused the following vocabularies 
\begin{itemize}
	\item GOLD\footnote{\url{http://linguistic-ontology.org}}, an ontology for descriptive linguistics.
	\item WGS84 Geo Basic Vocabulary\footnote{\url{http://www.w3.org/2003/01/geo/}}, for representing the geospatial data	 
\end{itemize} 

Nevertheless, it was necessary to create some own terms.

\subsection{Generation}
RDF has been generated within three main phases. Next we describe each one of them.

\begin{enumerate}
	\item Importing the spreadsheet into MySQL database by using the MySQL importing tool.
	\item Defining a set of R2RML\footnote{R2RML is a standard RDB2RDF mapping language \url{http://www.w3.org/TR/r2rml/}} mappings among the MySQL database and RDF vocabulary elements.
	\item Executing the R2RML engine, morph\footnote{\url{https://github.com/boricles/morph}}, for generating the RDF instances from the MySQL database data by using the defined R2RML mappings.  
\end{enumerate}


\subsection{Publication}
The generated RDF is stored in Virtuoso\footnote{\url{http://virtuoso.openlinksw.com/}} open source version, which
integrates with Pubby\footnote{\url{http://www4.wiwiss.fu-berlin.de/pubby/}} to publish the results, making
them available for humans and computers.

\subsection{Exploitation}
The resultant dataset exposes the linguistics resources enhanced with geospatial information, allowing for queries that otherwise require a lot of time by just looking at the original files. 


By extending the queries presented previously, we have built an application\footnote{\url{http://geo.linkeddata.es/map4rdf-dogon/}} for showing each of the Dogon villages on the map by using the tool map4rdf\footnote{\url{https://github.com/boricles/linked-data-visualization-tools}} \cite{deLeon_2012}. Figure \ref{fig:map4rdf} shows the application.

\begin{figure*}[htb!p]
\centering
%\includegraphics[scale=0.45]{IMGS/detalle_bn.png}
%\includegraphics[width=0.70\textwidth]{IMGS/detalle.pdf}
\includegraphics[width=0.99\textwidth]{img/map4rdf.png}
\caption{Visualization of the Dogon villages}
\label{fig:map4rdf}
\end{figure*}

%We then used the tool map4rdf\footnote{This can be accessed here: \url{http://code.google.com/p/map4rdf}} to map it upon a globe, with an online portal at: \url{http://oegdev.dia.fi.upm.es/projects/map4rdf} \cite{deLeon_2012} . At this stage, we have plotted each of the Dogon villages on the map. Each point also contains additional information about the language spoken in that village. The map is available here: \url{http://geo.linkeddata.es/map4rdf-dogon/#dashboard}


%I don't really know the details too much?

\section{Querying to Geospatial Information}
%I don't seem to have the query to get the code for each of the languages, and this is kind of repetitious to the other paper. What do you think? Should we include this? 


We also used RDF data gathered automatically from the World Atlas of Language Structures (WALS) \cite{Haspelmath_etal2008}. WALS can be freely accessed from \url{http://wals.info}, and the database can be downloaded as a .csv file. However, there is also an RDF endpoint, which has been integrated with the LLOD cloud. To access WALS data, then, we used the SPARQL endpoint available at: \url{http://mlode-sparql.nlp2rdf.org/sparql}.

In order to use the LLOD through the endpoint, go to the link above. Remove the `Default Data Set Name (Graph IRI)' in the top entry box by deleting \url{http://mlode.nlp2rd.org}, so that the box is empty. Write your query into the `Query Text' box, and press `Run Query.' The query will then load. The hyperlink of the loaded query can be used as a way to refer to that result without needing to re-enter the query each iteration. 

Using this method, we devised a SPARQL query that gathered the longitude and latitude of each language within WALS specified by the three-letter language identifier WALS uses. Using the same methods as above, we converted this data into RDF, and then plotted it with the map4rdf software. The results of this can be seen here: \url{http://geo.linkeddata.es/map4rdf-wals/#dashboard}. 

%Not at the moment it can't! 

For more details on the SPARQL endpoint for the LLOD, see Littauer {\it et al.}, this issue. 

\section{Conclusions}

We have briefly shown here a workflow to transform data from a simple spreadsheet into an RDF triple store that can queried using a SPARQL endpoint, and an application called map4rdf that uses this endpoint with GPS coordinates to visualize RDF data on a world map. Moreover, the tools that we have used here are open source and freely available. Converting linguistic data into RDF can be a straightforward process and we have shown the steps and some tools to assist in that transformation. There is much data available about languages and their typological features on the Web, which are often available in simple .csv formats. For example, the contents of World Atlas of Language Structures (WALS)\footnote{\url{http://wals.info}} \cite{Haspelmath_etal2008} have been converted from .csv to RDF and are available through the MLODE SPARQL endpoint.\footnote{\url{http://mlode-sparql.nlp2rdf.org/sparql}} It was a trivial task for us to set up map4rdf to point at the WALS RDF data, so that we could also visualize its contents, which contain over 2000 languages' data points. Whereas the online version of WALS already contains maps of typological features of languages, their use is limited and by leveraging RDF as we have with WALS and the Dogon data, we can easily combine these disparate datasets, so that, for example, we can merge data about languages and their typological features from both datasets. This allows us to visualize not only the villages where Dogon languages are spoken, but linguistic features of languages spoken in this area of Mali encoded in WALS. This mashup provides even more detailed information about the features of these different languages, which provides another important data source in untangling the mystery of why Dogon languages are so different than other language families in West Africa. It also shows the power of encoding data in RDF and leveraging RDF tools.


\bibliographystyle{IEEEtran}
\bibliography{bib.bib}

\end{document}

