


% In order to encourage future submission of resources into the LLOD, we present a new resource - a lexical and geospatial database of languages from the Dogon family, in Western Africa. According to the iterative and incremental Linked Data Life Cycle \cite{Villazon_2011}, all linked data follows a cycle: specification, modelling, generation, publication, and exploitation (which then feeds back into future specification). Here, we go through each of these stages on the way from taking the Dogon spreadsheet to an interactive visualisation using map4rdf. We also present a visualisation using a database already in the LLOD, the World Atlas of Language Structures (WALS) \cite{Haspelmath_etal2008}, by querying for geographic information for languages using a SPARQL endpoint for the LLOD (see Littauer {\it et al.}, this issue). We hope that these efforts will encourage future researchers to both add to and utilise the LLOD for their own research.  




%I'm note entirely sure of much related work, to be honest. Uhm. Yeah. Boris? Steve?
%% Shouldn't we be describing the Dogon dataset, somewhere?

In the visualization of language data, there has been work on displaying language differences on a broad scale, including presenting hierarchical and cross-linguistic data \cite{Rohrdantz:2012, RMB+10}, displaying related languages gathered from the World Atlas of Language Structures (WALS) by geographical proximity and relatedness \cite{LittEACL}, displaying word meanings on a map \cite{theron}, and displaying the location of languages that contain some type of typological feature language locations on a world map \cite{Haspelmath_etal2008}. There has also recently been visualizations that display language relatedness and dialectology using lexical items and location together \cite{10.1371/journal.pone.0023613}. 

In this work we derive RDF from simple table data stored in a spreadsheet, leverage the ability of RDF graphs to be easily merged, and harness different RDF tools to display geospatial data in the map4rdf software, which is freely available and runs in the browser. In doing so, we provide detailed information about the location of villages in Mali in which Dogon languages are spoken. Dogon is an interesting language family because until recently there was very little that was known about these languages. In fact, until as late as 1989, Dogon appeared in reference books on African languages as if it were a single language (cf.\ \citep{Bendor-Samuel_etal1989,Blench2005}). Current estimates from experts working in Mali is that there are now over 20 mutually unintelligble Dogon languages, with new varieties being ``discovered'' every year. However, the current genealogical relatedness of these languages is still unclear, as is the internal structure of the Dogon language family. Additionally, due to the typological characteristics of Dogon languages, such as these languages' lack of noun classes that are typical of sub-saharan West African languages in general or Dogon's SOV basic word order (instead of SVO like many of its neighbors), the position of the Dogon language family relative to other African language families remains unclear. Thus in disentangling the mysteries of how Dogon languages are related within their family, an interactive visual reference of where the languages are spoken is a useful tool for exploring avenues of possible genealogical decent due to geographic proximity and other effects like borrowing due to areal contact.
