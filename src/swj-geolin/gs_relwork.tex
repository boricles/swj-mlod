%I'm note entirely sure of much related work, to be honest. Uhm. Yeah. Boris? Steve?
%% Shouldn't we be describing the Dogon dataset, somewhere?

Regarding language visualisation, there has been some work on displaying language differences on a broad scale. This includes work presenting hierarchical or cross-linguistic data \cite{Rohrdantz:2012, RMB+10}, displaying related languages (gathered from WALS) by geographical proximity and relatedness \cite{LittEACL}, displaying word meaning with a world map \cite{theron}, and displaying language locations on a globe \cite{Haspelmath_etal2008}. However, the authors are not aware of any work using maps derived from data stored in RDF as here, although there have recently been more computational visualisations displaying language relatedness and dialectology using lexical items and location together \cite{10.1371/journal.pone.0023613}. 