
Linked Data presents many opportunities to access and share data in different formats and for different purposes. In linguistics and related fields like cultural archaeology and population genetics, visualization of data points on maps is particularly beneficial in formulating hypotheses about data sets, particularly sparse ones, which is often the case in these fields. In this short report, we describe how we converted a spreadsheet that contains information about endangered Dogon languages and where they are spoken in small rural villages in Mali, West Africa, into an Resource Description Framework (RDF) triple store so that we could leverage other RDF tools to visualize these data. The result gives researchers a clearer picture of the dispersal of Dogon speakers and neighboring languages and we show that the spreadsheet-to-RDF conversion pipeline that we develop is applicable to any data set that can be combined with GPS coordinates.
