
We have briefly shown here a workflow to transform data from a simple spreadsheet into an RDF triple store that can queried using a SPARQL endpoint, and an application called map4rdf that uses this endpoint with GPS coordinates to visualize RDF data on a world map. Moreover, the tools that we have used here are open source and freely available. Converting linguistic data into RDF can be a straightforward process and we have shown the steps and some tools to assist in that transformation. There is much data available about languages and their typological features on the Web, which are often available in simple .csv formats. For example, the contents of World Atlas of Language Structures (WALS)\footnote{\url{http://wals.info}} \cite{Haspelmath_etal2008} have been converted from .csv to RDF and are available through the MLODE SPARQL endpoint.\footnote{\url{http://mlode-sparql.nlp2rdf.org/sparql}} It was a trivial task for us to set up map4rdf to point at the WALS RDF data, so that we could also visualize its contents, which contain over 2000 languages' data points. Whereas the online version of WALS already contains maps of typological features of languages, their use is limited and by leveraging RDF as we have with WALS and the Dogon data, we can easily combine these disparate datasets, so that, for example, we can merge data about languages and their typological features from both datasets. This allows us to visualize not only the villages where Dogon languages are spoken, but linguistic features of languages spoken in this area of Mali encoded in WALS. This mashup provides even more detailed information about the features of these different languages, which provides another important data source in untangling the mystery of why Dogon languages are so different than other language families in West Africa. It also shows the power of encoding data in RDF and leveraging RDF tools.
