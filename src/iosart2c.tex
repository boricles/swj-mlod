% Journal:
%   Journal of Ambient Intelligence and Smart Environments (JAISE), IOS Press
%   Web Intelligence and Agent Systems: An International Journal (wias)
%   Semantic Web: Interoperability, Usability, Applicability (SW)
% Latex 2e
% Test file iosart2c.tex

%[seceqn,secfloat,secthm,crcready]

% options: wias, jaise, sw
\documentclass{iosart2c}


\usepackage[T1]{fontenc}
\usepackage{times}%

\usepackage{natbib}
%\usepackage[dvips]{hyperref}
\usepackage{amsmath}
\usepackage{dcolumn}
%\usepackage{endnotes}
\usepackage{graphics}


\newcolumntype{d}[1]{D{.}{.}{#1}}


\firstpage{1} \lastpage{5} \volume{1} \pubyear{2009}


\begin{document}
\begin{frontmatter}                           % The preamble begins here.

% CHANGE AS SUITED - RL.
%\pretitle{Pretitle}
\title{Using the LLOD cloud for Querying Linguistic Resources and Geospatial Mapping %\thanks{Footnote in title.}
}

%Linguistic resources enhanced with geospatial Information
%Query for identifying all linguistic resources in the LLOD by ISO 639-3


\runningtitle{Using the LLOD cloud for Querying Linguistic Resources and Geospatial Mapping}
%\subtitle{Subtitle}

%\review{Name Surname, University, Country}{Name Surname, University, Country}{Name Surname, University, Country}


\author[A]{\fnms{Steven} \snm{Moran}%\thanks{Corresponding author. E-mail: editorial@iospress.nl.}%\thanks{Do not use capitals for the author's surname.}
},
\author[B,C]{\fnms{Richard} \snm{Littauer}}
and
\author[D]{\fnms{Boris} \snm{Villazon-Terrazas}}
\runningauthor{Moran, Littauer and Villazon-Terrazas}
\address[A]{Research Unit Quantitative Language Comparison, Ludwig Maximilian University, Geschwister Scholl Platz 1, D-80539 Munich, Germany\\ 
E-mail: bambooforest@gmail.com} %Steve? Probably not the right email. 


%\address[A]{Journal Production Department, IOS Press, Nieuwe Hemweg 6b, 1013 BG, Amsterdam,\\ The Netherlands\\ E-mail: first@somewhere.com}
%\address[B]{Department first, then University or Company name, Insert a complete correspondence (mailing) address, Abbreviate US states, Include country\\ E-mail: \{second,third\}@somewhere.com}
\address[B]{Department of Intelligent Computer Systems, University of Malta, Msida, MSD2080, Malta}
\address[C]{Computational Linguistics Department, Saarland University, Saarbr\"ucken, 66121, Germany\\  E-mail: littauer@coli.uni-saarland.de}
\address[D]{Intelligent Software Components, iSOCO, S.A., Av. del Partenon 16-18, Madrid, Spain\\
E-mail: bvillazon@isoco.com}

\begin{abstract}
We present two uses of the Linguistics Linked Open Data (LLOD) cloud. First, we present the SPARQL endpoint for the LLOD, the largest linked data cloud for open linguistics currently available. We showcase how to query for language resources within the cloud, with example queries. Second, we present a way to visualise geospatial information, either by using a self-maintained spreadsheet and map4rdf, or by querying the LLOD. We present two examples; first, a spreadsheet with lexical and geospatial information for the Dogon languages of West Africa, and secondly, query results for languages with geospatial information listed in the World Atlas of Language Structure. It is hoped that the work presented here will expedite LLOD use by researchers. %117 words at present.

%The abstract should be clear, descriptive, self-explanatory and no longer than 200 words. It should alsobe suitable for publication in abstracting services. Do not include references or formulae in the abstract.
\end{abstract}

\begin{keyword}
Semantic Web\sep Linked Data\sep LLOD\sep Linguistics\sep Typology\sep Language Resources\sep Geospatial Mapping
%Keyword one\sep keyword two\sep keyword three\sep keyword four\sep keyword five
\end{keyword}

\end{frontmatter}


\section{Introduction}
%
%Richard's hackday project aimed at querying the LLOD via ISO 639-3
%code. At that time, these queries worked (see below for several of
%them). Now they don't seem to return anything (hence I haven't
%included them in the MLODE hack day results). There might be some
%fairly obvious reason for them not working, but I'm not sure what it
%is. I do remember that at one point the URL or the endpoint or
%something was changed for some reasons (hosting was with Pablo's
%group?).
%I do think though that the ability to query LLOD for resources via
%language code, etc., is a very, very strong selling point for the data
%post-proceedings and for the LLOD cloud in general. We aren't going to
%get any serious users (do we currently have any users?) beyond the
%Leipzig group (and perhaps some others), if it's not functioning (or
%if we can't explain how to use it since what was once working no
%longer works).
%
%It would be great to get such "exploratory" data queries working and I
%know Richard and a few others want to submit something to the data
%post-proceedings on exactly that.

%ENDPOINT: http://mlode-sparql.nlp2rdf.org/sparql

The Linguistics 

\section{Related Work}
\input{relatedwork}

\section{Querying the LLOD for language resources}
%%
%%Richard Littauer, Didier Cherix and Steven Moran
%%http://mlode.nlp2rdf.org/sparql
%%
%%
%%Using the SPARQL endpoint (above), this team set itself the goal of devising SPARQL queries to identify all resources in the LLOD that have data with regard to a given ISO 639-3 unique language name identifier.
%
%%http://mlode.nlp2rdf.org/sparql?default-graph-uri=&query=prefix+wals%3A+%3Chttp%3A%2F%2Fwals.info%2Flanguage%2F%3E%0D%0Aselect+distinct+%3Frelation+where+%7B%0D%0Awals%3Achr+%3Chttp%3A%2F%2Fpurl.org%2Fdc%2Fterms%2Frelation%3E+%3Frelation+.%0D%0A%7D&format=text%2Fhtml&timeout=0&debug=on
%
%And the other ones are:
%
%1. goto: http://mlode.nlp2rdf.org/sparql
%
%2. remove the default text in "Default Data Set Name (Graph IRI)"
%
%3. paste this query:
%
%prefix wals: <http://wals.info/language/>
%select distinct  ?label ?descr ?ref ?area ?lat ?long ?genus
%where
%{
%?s <http://purl.org/dc/terms/subject> wals:chr .
%?s <http://wals.info/vocabulary/hasValue> ?value .
%?value <http://purl.org/dc/terms/description> ?descr .
%wals:chr <http://www.w3.org/2003/01/geo/wgs84_pos#lat> ?lat .
%wals:chr <http://www.w3.org/2003/01/geo/wgs84_pos#long> ?long .
%
%wals:chr ?feature ?datapoint .
%wals:chr rdfs:label ?label .
%?datapoint <http://purl.org/dc/terms/references> ?ref .
%
%?feature <http://purl.org/dc/terms/isPartOf> ?chapter .
%?chapter <http://wals.info/vocabulary/chapterArea> ?area .
%wals:chr <http://wals.info/vocabulary/hasGenus> ?genus.
%wals:chr <http://wals.info/vocabulary/altName> ?name .
%}
%
%4. hit "Run query"
%
%And this query:
%
%run steps 1, 2 and for 3 paste this into the text box:
%
%prefix wals: <http://wals.info/language/>
%select distinct ?relation where {
%wals:chr <http://purl.org/dc/terms/relation> ?relation .
%}
%
%and click on "Run query"
%
%The second query shows us every resource that contains information
%about a given ISO 639-3 code in the LLOD cloud. Could you also get the
%same type of thing via querying via a WALS language code, getting the
%its ISO code and then running the query above.
%%%%NEED TO DO THIS. _RICHARD
%
%For the first query, you simply return all and any information in the
%WALS given a WALS language code (wals:chr in this example). Correct?


%First query URL:
%
%http://mlode.nlp2rdf.org/sparql?default-graph-uri=&query=prefix+wals%3A+%3Chttp%3A%2F%2Fwals.info%2Flanguage%2F%3E%0D%0Aselect+distinct++%3Flabel+%3Fdescr+%3Fref+%3Farea+%3Flat+%3Flong+%3Fgenus%0D%0Awhere%0D%0A%7B%0D%0A%3Fs+%3Chttp%3A%2F%2Fpurl.org%2Fdc%2Fterms%2Fsubject%3E+wals%3Achr+.%0D%0A%3Fs+%3Chttp%3A%2F%2Fwals.info%2Fvocabulary%2FhasValue%3E+%3Fvalue+.%0D%0A%3Fvalue+%3Chttp%3A%2F%2Fpurl.org%2Fdc%2Fterms%2Fdescription%3E+%3Fdescr+.%0D%0Awals%3Achr+%3Chttp%3A%2F%2Fwww.w3.org%2F2003%2F01%2Fgeo%2Fwgs84_pos%23lat%3E+%3Flat+.%0D%0Awals%3Achr+%3Chttp%3A%2F%2Fwww.w3.org%2F2003%2F01%2Fgeo%2Fwgs84_pos%23long%3E+%3Flong+.%0D%0A%0D%0Awals%3Achr+%3Ffeature+%3Fdatapoint+.%0D%0Awals%3Achr+rdfs%3Alabel+%3Flabel+.%0D%0A%3Fdatapoint+%3Chttp%3A%2F%2Fpurl.org%2Fdc%2Fterms%2Freferences%3E+%3Fref+.%0D%0A%0D%0A%3Ffeature+%3Chttp%3A%2F%2Fpurl.org%2Fdc%2Fterms%2FisPartOf%3E+%3Fchapter+.%0D%0A%3Fchapter+%3Chttp%3A%2F%2Fwals.info%2Fvocabulary%2FchapterArea%3E+%3Farea+.%0D%0Awals%3Achr+%3Chttp%3A%2F%2Fwals.info%2Fvocabulary%2FhasGenus%3E+%3Fgenus.%0D%0Awals%3Achr+%3Chttp%3A%2F%2Fwals.info%2Fvocabulary%2FaltName%3E+%3Fname+.%0D%0A%7D&format=text%2Fhtml&timeout=0&debug=on
%
%Second query URL:
%
%http://mlode.nlp2rdf.org/sparql?default-graph-uri=&query=prefix+wals%3A+%3Chttp%3A%2F%2Fwals.info%2Flanguage%2F%3E%0D%0Aselect+distinct+%3Frelation+where+%7B%0D%0Awals%3Achr+%3Chttp%3A%2F%2Fpurl.org%2Fdc%2Fterms%2Frelation%3E+%3Frelation+.%0D%0A%7D&format=text%2Fhtml&timeout=0&debug=on

%%%%%%%%%%%%%%%%%%

%A link to the endpoint with a query for Chrua, a Vietnamese language:
%
%http://mlode-sparql.nlp2rdf.org/sparql?default-graph-uri=&query=prefix+wals%3A+%3Chttp%3A%2F%2Fwals.info%2Flanguage%2F%3E%0D%0Aselect+distinct+%3Frelation+where+%7B%0D%0Awals%3Achr+%3Chttp%3A%2F%2Fpurl.org%2Fdc%2Fterms%2Frelation%3E+%3Frelation+.%0D%0A%7D&format=text%2Fhtml&timeout=0&debug=on
%
%A SPARQL query for all of the WALS information for a given ISO 639-3:
%
%prefix wals: <http://wals.info/language/>
%
%select distinct  ?label ?descr ?ref ?area ?lat ?long ?genus where
%{
%?s <http://purl.org/dc/terms/subject> wals:chr .
%?s <http://wals.info/vocabulary/hasValue> ?value .
%?value <http://purl.org/dc/terms/description> ?descr .
%wals:chr <http://www.w3.org/2003/01/geo/wgs84_pos#lat> ?lat .
%wals:chr <http://www.w3.org/2003/01/geo/wgs84_pos#long> ?long .
%
%wals:chr ?feature ?datapoint .
%wals:chr rdfs:label ?label .
%?datapoint <http://purl.org/dc/terms/references> ?ref .
%
%?feature <http://purl.org/dc/terms/isPartOf> ?chapter .
%?chapter <http://wals.info/vocabulary/chapterArea> ?area .
%wals:chr <http://wals.info/vocabulary/hasGenus> ?genus.
%wals:chr <http://wals.info/vocabulary/altName> ?name .
%}
%
%A SPARQL query to list all of the resources in the LLOD for the
%specific ISO 639-3 code:
%
%prefix wals: <http://wals.info/language/>
%select distinct ?relation where {
%wals:chr <http://purl.org/dc/terms/relation> ?relation .
%}
%
%In order to access this code, one needs to uncheck the 'default data
%set' by removing the link:
%http://mlode.nlp2rdf.org
%
%From the first entry box at the endpoint:
%
%http://mlode-sparql.nlp2rdf.org/sparql


\section{Language resources with geospatial information}
%%Steven Moran, Richard Littauer, Didier Cherex and Boris Villaz�n-Terrazas
%http://www.slideshare.net/boricles/linguistic-resources-enhanced-with-geospatial-information
%
%This team converted a dataset in spreadsheet format that contains lexical data and GPS coordinates from Dogon languages spoken in Mali, West Africa and converted these data into RDF. They then used the tool map4rdf
%http://code.google.com/p/map4rdf/ 
%
%http://oegdev.dia.fi.upm.es/projects/map4rdf/
%
%to plot Dogon villages on the map. Each point contains additional information about the language spoken in that village. The map is available here:
%http://geo.linkeddata.es/map4rdf-dogon/#dashboard
%
%The team also used the RDF data of The World Atlas of Language Structures (WALS) and plotted with map4rdf software:
%http://wals.info/
%
%http://geo.linkeddata.es/map4rdf-wals/#dashboard




\section{Conclusions}



\end{document}
