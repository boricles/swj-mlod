% Journal:
%   Journal of Ambient Intelligence and Smart Environments (JAISE), IOS Press
%   Web Intelligence and Agent Systems: An International Journal (wias)
%   Semantic Web: Interoperability, Usability, Applicability (SW)
% Latex 2e
% Test file iosart2c.tex

%[seceqn,secfloat,secthm,crcready]

% options: wias, jaise, sw
\documentclass{iosart2c}


\usepackage[T1]{fontenc}
\usepackage{times}%

\usepackage{natbib}
%\usepackage[dvips]{hyperref}
\usepackage{amsmath}
\usepackage{dcolumn}
%\usepackage{endnotes}
\usepackage{graphics}


\newcolumntype{d}[1]{D{.}{.}{#1}}


\firstpage{1} \lastpage{5} \volume{1} \pubyear{2009}


\begin{document}
\begin{frontmatter}                           % The preamble begins here.

%
%\pretitle{Pretitle}
\title{MLODE paper \thanks{Footnote in title.}}

\runningtitle{MLODE paper}
%\subtitle{Subtitle}

%\review{Name Surname, University, Country}{Name Surname, University, Country}{Name Surname, University, Country}


\author[A]{\fnms{Steven} \snm{Moran}\thanks{Corresponding author. E-mail: editorial@iospress.nl.}\thanks{Do not use capitals for the author's surname.}},
\author[B]{\fnms{Richard} \snm{Littauer}}
and
\author[C]{\fnms{Boris} \snm{Villazon-Terrazas}}
\runningauthor{Moran, Littauer and Villazon-Terrazas}
\address[A]{Journal Production Department, IOS Press, Nieuwe Hemweg 6b, 1013 BG, Amsterdam,\\ The Netherlands\\
E-mail: first@somewhere.com}
\address[B]{Department first, then University or Company name, Insert a complete correspondence (mailing) address,
Abbreviate US states, Include country\\
E-mail: \{second,third\}@somewhere.com}
\address[C]{Intelligent Software Components, iSOCO, S.A., Av. del Partenon 16-18, Madrid, Spain\\
E-mail: \{bvillazon\}@isoco.com}

\begin{abstract}
The abstract should be clear, descriptive, self-explanatory and no longer than 200 words. It should also
be suitable for publication in abstracting services. Do not include references or formulae in the abstract.
\end{abstract}

\begin{keyword}
Keyword one\sep keyword two\sep keyword three\sep keyword four\sep keyword five
\end{keyword}

\end{frontmatter}


\section{Introduction}
%
%Richard's hackday project aimed at querying the LLOD via ISO 639-3
%code. At that time, these queries worked (see below for several of
%them). Now they don't seem to return anything (hence I haven't
%included them in the MLODE hack day results). There might be some
%fairly obvious reason for them not working, but I'm not sure what it
%is. I do remember that at one point the URL or the endpoint or
%something was changed for some reasons (hosting was with Pablo's
%group?).
%I do think though that the ability to query LLOD for resources via
%language code, etc., is a very, very strong selling point for the data
%post-proceedings and for the LLOD cloud in general. We aren't going to
%get any serious users (do we currently have any users?) beyond the
%Leipzig group (and perhaps some others), if it's not functioning (or
%if we can't explain how to use it since what was once working no
%longer works).
%
%It would be great to get such "exploratory" data queries working and I
%know Richard and a few others want to submit something to the data
%post-proceedings on exactly that.

%ENDPOINT: http://mlode-sparql.nlp2rdf.org/sparql

The Linguistics 

\section{Related Work}
\input{relatedwork}

\section{Language resources with geospatial information}
%%Steven Moran, Richard Littauer, Didier Cherex and Boris Villaz�n-Terrazas
%http://www.slideshare.net/boricles/linguistic-resources-enhanced-with-geospatial-information
%
%This team converted a dataset in spreadsheet format that contains lexical data and GPS coordinates from Dogon languages spoken in Mali, West Africa and converted these data into RDF. They then used the tool map4rdf
%http://code.google.com/p/map4rdf/ 
%
%http://oegdev.dia.fi.upm.es/projects/map4rdf/
%
%to plot Dogon villages on the map. Each point contains additional information about the language spoken in that village. The map is available here:
%http://geo.linkeddata.es/map4rdf-dogon/#dashboard
%
%The team also used the RDF data of The World Atlas of Language Structures (WALS) and plotted with map4rdf software:
%http://wals.info/
%
%http://geo.linkeddata.es/map4rdf-wals/#dashboard


\section{Conclusions}
\input{conclusions}

\end{document}
